\documentclass[12pt,oneside]{amsart}

%\documentclass{article}

\usepackage{amsthm,amsmath,amssymb}
\usepackage[numbers]{natbib}

%\usepackage[colorlinks]{hyperref}
\usepackage{hyperref}
\hypersetup{
    colorlinks,%
    citecolor=black,%
    filecolor=black,%
    linkcolor=black,%
    urlcolor=black
}
\usepackage{hypernat}
\usepackage[top=2.5cm, bottom=2.5cm, left=2.8cm,
right=2.8cm]{geometry} 

%\setlength{\parskip}{0pt}
%\setlength{\parsep}{0pt}
%\setlength{\headsep}{0pt}
%\setlength{\topskip}{0pt}
%\setlength{\topmargin}{0pt}
%\setlength{\topsep}{0pt}
%\setlength{\partopsep}{0pt}

\linespread{1}

%\usepackage{marvosym}

%\textwidth = 6.5 in \textheight = 9.2 in \oddsidemargin = 0.0 in
%\evensidemargin = 0.0 in \topmargin = -0.0 in \headheight = 0.0 in
%\headsep = 0.0in \parskip = 0.0in \parindent = 0.0in
%\def\baselinestretch{0.871}


\newtheorem{theorem}{Theorem}[section]
\newtheorem{proposition}[theorem]{Proposition}
\newtheorem{lemma}[theorem]{Lemma}
\newtheorem{defn}[theorem]{Definition}
\newtheorem{corollary}[theorem]{Corollary}
\newtheorem{remark}{Remark}[section]
\newtheorem{example}[theorem]{Example}

\newcommand{\E}{{\mathbb E}}
\newcommand{\se}{{\mathcal{E}}}
\newcommand{\R}{{\mathbb R}}
\renewcommand{\P}{{\mathbb P}}
\newcommand{\ac}{{\mathcal{A}}}
\newcommand{\A}{{\mathcal{A}}}
\newcommand{\B}{{\mathcal{B}}}
\newcommand{\C}{{\mathcal{C}}}
\newcommand{\T}{{\mathcal{T}}}
\newcommand{\Z}{{\mathcal{Z}}}
\newcommand{\K}{{\mathcal{K}}}
\newcommand{\lc}{{\mathcal{L}}}
\newcommand{\Ps}{{\mathcal{P}}}
\newcommand{\G}{{\mathcal{G}}}
\newcommand{\pa}{{\mathring{p}}}
\newcommand{\F}{{\cal F}}
\newcommand{\samp}{{\mathcal{X}}}
\newcommand{\X}{{\mathcal{X}}}
\newcommand{\hi}{{\hat{\Phi}}}


\newcounter{rcnt}[section]
\renewcommand{\thercnt}{(\roman{rcnt})}

\def\qt#1{\qquad\text{#1}}

\def\argmin{\mathop{\rm argmin}}
\def\argmax{\mathop{\rm argmax}}


\setlength{\parskip}{1.4 \medskipamount}
%\sloppy
%\linespread{1.3}

\begin{document}

\title{STAT 238: Bayesian Statistics}
\author{Instructor: Aditya Guntuboyina \\ Course Outline for Spring
  2026 \\ University of California, Berkeley}  

\maketitle

\begin{itemize}
\item \textbf{Instructor}: Aditya Guntuboyina. Email:
  \texttt{aditya@stat.berkeley.edu} and Website: 
  \url{www.stat.berkeley.edu/~aditya}
\item \textbf{Lectures}: 1 pm to 1:59 pm on Mondays, Wednesdays and
  Fridays in 9 Lewis Hall. 
\item \textbf{Office Hours}: 2-3 pm on Mondays and
  Fridays in 443 Evans Hall. 
\item \textbf{GSI}: Reece D Huff. Email: \texttt{rdhuff@berkeley.edu}
\item \textbf{GSI Lab Section}: 9 am - 10:59 am  or 12 pm -
  1:59 pm in 340 Evans Hall on Tuesdays  
\item \textbf{GSI Office Hours}: 3-5 pm on Tuesdays and 11 am - 12 pm
  on Wednesdays in 444 Evans Hall
\end{itemize}

\noindent \textbf{About the course}: This course develops Bayesian
statistics from first principles of probability, emphasizing
interpretation, coherence, and contrasts with frequentist inference,
then moves through core parametric estimation and testing problems
(including empirical Bayes), linear and generalized linear models,
model selection, and high-dimensional regression. It introduces
nonparametric regression via Gaussian processes and kernels, with
connections to RKHS theory and applications such as Bayesian
optimization. A substantial component focuses on Bayesian
computation—Monte Carlo methods, MCMC (including Langevin and
Hamiltonian dynamics), variational inference, and modern
neural-network–enhanced sampling and approximation techniques. The
course culminates in advanced applications, including hierarchical and
multilevel models, sparse and variable-selection methods, mixture and
latent variable models, Bayesian neural networks and trees,
probabilistic numerics, optimization, and inverse problems. 

\noindent \textbf{Prerequisites}: Undergraduate probability at the
level of STAT 134, DATA 140, or EE 126 is required. In addition,
undergraduate statistics at the level of STAT 133 and STAT 135 is
required. 

\noindent \textbf{Programming Language}: You are free to use any
language (e.g., R, Python, Julia, Matlab etc.) for working on your
homework. We will be using Python code in class and the lab sections. 

\noindent \textbf{Text}: There is no required textbook for the
class. A variety of references will be provided, and you are welcome
to consult any of them. 

I will provide materials for each lecture (including slides or typed
lecture notes, and code) which will be posted \textit{after} the
lecture. 

\noindent \textbf{Ed Discussion}: I have created a site
for this class at Ed Discussion and we will use this platform for Q \&
A.

\noindent \textbf{Homework assignments}: There will be six homework
assignments. Your lowest homework score will be dropped when
calculating the final grade, and all remaining homework assignments
will be weighted equally. 

The homework assignments Will be posted on bcourses
according to the following schedule. Solutions will need to be
uploaded on Gradescope. 
\begin{itemize}
\item Homework One - will be posted on Jan 30 and due on Feb 13
\item Homework Two - will be posted on Feb 13 and due on Feb 27
\item Homework Three - will be posted on Feb 27 and due on March 13
\item Homework Four - will be posted on March 13 and due on April 03
\item Homework Five - will be posted on April 03 and due on April 17
\item Homework Six - will be posted on April 17 and due on May 01  
\end{itemize}
You have a total of 120 late hours that you can apply to your homework
for the entire semester. No points will be awarded for any homework
which brings the total late hours to more than 120.   

\noindent \textbf{Final Project}: There will be a final project. I
will post a list of suggested topics by April 1. Each student will
work individually on one topic (students may propose a topic outside
the list with prior approval). 

Each student is required to record a 20–25 minute presentation
explaining their project and findings, and to submit a Jupyter
notebook that fully reproduces all numerical analyses. Both the
presentation video and the notebook must be uploaded to bCourses by
11:59 PM on May 11. 

\noindent \textbf{Assessment}: Your final score for the class will be
calculated as
\begin{equation*}
  50 \%~ \text{Homework} + 50 \% ~ 
  \text{Final Project}.  
\end{equation*}
Each homework assignment is worth an equal amount (we will drop the
lowest homework score).  


\noindent \textbf{Grade Complaints}: If you have a complaint
against an assigned homework or project grade and want to talk to me
about it, first send me a written request through email explaining
your case clearly.

\noindent \textbf{Academic Integrity}: Collaboration in small groups
on homework problems is encouraged, but your write-up must be entirely
your own, and you may not read or copy another student’s
solutions. You may consult books, online resources, and generative AI
tools (e.g., ChatGPT or other LLMs) to support your learning; if you
do, you must (i) acknowledge all such sources/tools in your write-up
and (ii) ensure you fully understand and can independently reproduce
and explain every step of your solution. Copying text, code, or
solutions—verbatim or paraphrased—from any source (including AI tools)
without clear attribution, or submitting work you do not understand,
is cheating. Students found to be cheating risk failing the course and
being referred to the Office of Student Conduct. 

How to cite AI briefly (example): “Consulted ChatGPT for hints
on Problem 4 (a)” 

\noindent \textbf{Students with disabilities}: If you need
accommodations for any physical, psychological, or learning
disability, please get in touch with me so that we can make the
necessary arrangements. 





\end{document}

%%% Local Variables:
%%% mode: latex
%%% TeX-master: t
%%% End:
